\chapter{Simulations}
\label{chap:simulation}
To illustrate the working principle of IDI and to examine the Signal-to-Noise  characteristics, different simulations were performed:

First, it was assumed that the object to be imaged consists of discrete emitters, each emitting monochromatic spherical waves with the same wavelength, but with a randomly chosen phase, and the speckle image on a pixelated detector was simulated by addition of the scalar electric fields and taking the squared magnitude for each pixel. To reduce the influence of this discrete sampling on the simulated speckle patterns, the simulation is performed at XX the resolution and downsampled, such that each data point is the result of 4x4 discrete calculations
In this configuration, the speckle images of a single particle with randomly positioned emitters inside (approximating a single particle imaging setup), a focal volume filled with randomly positioned (non intersecting) hard spheres consisting of randomly positioned atoms (approximating for example many spherical nano particles imaged simultaneously) as well as a crystalline structure with emitters positioned at within a lattice were simulated. In the first two cases, a small-angle regime was chosen and the reconstruction was performed in 2D and as a 1D radial profile. For the crystalline structure, a realistic lattice constant in the same order of magnitude as the K$\alpha$ wavelength moves the reconstruction out of the small-angle regime and a 3D reconstruction of the reciprocal space was performed.
Additionally, in these simulations the effect of under-sampling was studied.

Second, to examine the influence of the fluorescence lifetime and pulse width, a time-resolved simulation was performed.
The results were compared with the approximation, that the contrast is determined by the product of the different number of modes.

Ultimately, the results of these simulations were used the determine the feasibility of an experimental setup using IDI
\section{Time independent Simulations}
In an infinite coherence time, stationary sources approximation, the simulation of the speckle pattern can be performed time independently the superposition of scaler electrical fields emitted with random phases. The simulation of the intensity at multiple discrete points can be performed in parallel using GPU acceleration, resulting in a simple and fast to evaluate model.
\subsection{Single Sphere}
\subsection{Multiple Spheres}
\subsection{Crystal}
\section{Time dependent Simulations}

Each of the $N$ atoms is assigned an emitting time $t_{n}$ chosen according to the excitation pulse shape and its position. Starting from this emitting time, the atom emits an exponential decaying field with a decay time chosen to match the lifetime. 


For each discrete pixel on the simulated detector, for each atom the distance $d_n$, the arrival time $t'_n=t_n+d_n/c$ of each atom's initial radiance, and its time independent complex field $E_n=\frac{1}{d} e^{ikd_n+\phi_n}$ with initial random phase $phi$ is calculated.
The time dependent E field is the summation over the decaying field of all atoms,
\begin{equation}
E(t)=\sum_{n=0}^N  E_n \Theta(t'_n  - t) * e^{-(t-t'_n )/\tau}
\label{eq:tdsum}
\end{equation}
and the simulated intensity the time integral over the magnitude squared of the E-field,
\begin{equation}
I=\int_0^\infty \left| E(t) \right|^2 .
\end{equation}

To efficiently solve this integral for each detector pixel, it can be splitted into N parts with a constant number atoms which radiations has already arrived and each of those parts can be solved analytically. For this, first all atoms are sorted by the arrival time at the pixel. At each arrival time $t_n$, the sum in \fref{eq:tdsum} gets a new term and the field is calculated as
\begin{equation}
E(t_n)=E(t_{n-1})*e^{\sfrac{t_{n-1}-t_n}{\tau}}+E_n
\end{equation}
 This can be done with a parallel inclusive scan, as shown in \fref{algo:td}. This gives the supports for the integral, as show in \fref{fig:tdint}, which can now be solved as
\begin{equation}
	I=\int_0^\infty \left| E(t) \right|^2 = \sum
\end{equation}


This procedure is efficient in regards of discrete time steps that need to be calculated and can easily be parallelized using a GPU.


\begin{algorithm}
	\caption{Time dependent Simulation}\label{timesim}
	\begin{algorithmic}
	\Procedure{Scan}{$x \in \mathbb{C}^N$, $t \in \mathbb{R}^N$, $\tau\ \in \mathbb{R}$}
	\Comment{Exponentially decaying inclusive prefix sum / scan} 
		\State $s \gets 1$
		\While{$s<N$} 
		\Comment{Scan Upsweep}
			\For{k  $\gets 0$ to $N-1$ step $2s$ \textbf{parallel}}
				\State $decay \gets exp\left(\frac{t[k+s-1]-t[k+2s-1]}{\tau}\right)$ 
				\State $x[k+2s-1] \gets x[k+2s-1] + decay*x[k+s-1]$				
			\EndFor
			\State $s \gets 2s$ 
		\EndWhile
		\State $s \gets N/2$
		\While{$s>1$}  
		 \Comment{Scan Downsweep}
			\State $s \gets s/2$
			\For{k  $\gets 0$ to $N-1-2s$ step $2s$ \textbf{parallel}}
				\State $decay \gets exp\left(\frac{t[k+2s-1]-t[k+3s-1]}{\tau}\right)$
				\State $x[k+3s-1] \gets x[k+3s-1] + decay*x[k+2s-1]$
			\EndFor
		\EndWhile
	\EndProcedure
	
	\Function{Simulation}{Atom positions $x \in \mathbb{R}^{Nx3}$,  Detector position $y \in \mathbb{R}^{3}$, \newline Initial Phases $\phi \in [0,2\pi)^N$, Emission Times $t_0 \in \mathbb{R}^N$, $\tau\ \in \mathbb{R}$}
	\State	\Call{Prepare}{}
	\State	\Call{Sort}{($a$,$t$) by $t$}
	\State 	\Call{Scan}{$a$, $t$, $\tau$}
	\State $Result \gets \sum_{n=0}^N a_n$
	\State \Return $Result$
	\EndFunction
	
	\end{algorithmic}
\label{algo:td}
\end{algorithm}

\begin{figure}
	\begin{subfigure}[b]{0.45\textwidth}
		\includegraphics[width=\linewidth]{images/tdsphere.pdf}
		\caption{Reconstructed radial profiles at different pulse FWHM and fixed decay time $\tau = 0.1$\,fs}
	\end{subfigure}
	\begin{subfigure}[b]{0.45\textwidth}
		\includegraphics[width=\linewidth]{images/tdpspherevis.pdf}
		\caption{Visibility in the reconstruction for different pulse FWHM and decay times $\tau$}
	\end{subfigure}
	\caption[Time Dependent IDI Simulation of a Sphere]{Time Dependent IDI Simulation of a Sphere: For a sphere with 10\,nm radius consisting of $2*10^5$ atoms emitting 6.4\,keV fluorescence captured by an 256x256@50\,um detector in 20\,cm distance, a series of simulations were performed with different decay times $\tau$ of the emission and different exciting pulse FWHM. In a) exemplary radial profiles of the reconstruction are shown for one fixed $\tau$. Those reconstructions are used to plot the  dependence of the visibility on the pulse width in b). For long pulses, an $1/x$ relation is visible.}
	\label{fig:tdpshere}
\end{figure}

\subsection{Signal to Noise}

\section{Implications for an experimental design}
