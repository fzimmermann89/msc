\chapter{Simulations}
\label{chap:simulation}
To illustrate the working principle of IDI and to examine the Signal-to-Noise  characteristics, different simulations were performed:

First, it was assumed that the object to be imaged consists of discrete emitters, each emitting monochromatic spherical waves with the same wavelength, but with a randomly chosen phase, and the speckle image on a pixelated detector was simulated by addition of the scalar electric fields and taking the squared magnitude for each pixel. To reduce the influence of this discrete sampling on the simulated speckle patterns, the simulation is performed at XX the resolution and downsampled, such that each data point is the result of 4x4 discrete calculations
In this configuration, the speckle images of a single particle with randomly positioned emitters inside (approximating a single particle imaging setup), a focal volume filled with randomly positioned (non intersecting) hard spheres consisting of randomly positioned atoms (approximating for example many spherical nano particles imaged simultaneously) as well as a crystalline structure with emitters positioned at within a lattice were simulated. In the first two cases, a small-angle regime was chosen and the reconstruction was performed in 2D and as a 1D radial profile. For the crystalline structure, a realistic lattice constant in the same order of magnitude as the K$\alpha$ wavelength moves the reconstruction out of the small-angle regime and a 3D reconstruction of the reciprocal space was performed.
Additionally, in these simulations the effect of under-sampling was studied.

Second, to examine the influence of the fluorescence lifetime and pulse width, a time-resolved simulation was performed.
The results were compared with the approximation, that the contrast is determined by the product of the different number of modes.

Ultimately, the results of these simulations were used the determine the feasibility of an experimental setup using IDI
\section{Time independent Simulations}
In an infinite coherence time, stationary sources approximation, the simulation of the speckle pattern can be performed time independently the superposition of scaler electrical fields emitted with random phases. The simulation of the intensity at multiple discrete points can be performed in parallel using GPU acceleration, resulting in a simple and fast to evaluate model.
\subsection{Single Sphere}
\subsection{Multiple Spheres}
\subsection{Crystal}
\section{Time dependent Simulations}

Each atoms is assigned an emitting time  chosen according to the excitation pulse shape. Starting from this emitting time  $t_{a}$, the atom emits and exponential decaying field with a decay constant chosen to match the lifetime. 
For each discrete pixel on the simulated detector, the arrival time of each atom's initial radiance is calculated and the atoms are sorted by their arrival time on the detector. To calculate the time dependent E-field, the following equation is used:
\begin{equation}
E(t)=\sum_{atoms}  E(atom) \Theta(t_{arrival}(atom)  - t) * e^{-a(t-t_{arrival}(atom) )}
\end{equation}



with $E(atom) $ the time independent complex E-field.


The intensity in one pixel can now be calculated as the time integral over the magnitude squared of the E-field.
This integral can be splitted into N parts with a constant number atoms which radiations has already arrived and each of those parts can be solved analytically. 



This procedure is efficient in regards of discrete time steps that need to be calculated.

If the atoms are sorted by their arrival time, for each of the integration parts, 

\begin{algorithm}
	\caption{Decaying prefix sum}\label{prefix}
	\begin{algorithmic}
	\Procedure{upsweep}{$x \in \mathbb{C}^N$, $t \in \mathbb{R}^N$, $\tau\ \in \mathbb{R}$} \Comment{first pass with increasing step size}
		\State $s \gets 1$
		\While{$s<N$}    \Comment{ O(log N)}
			
			\For{k  $\gets 0$ to $N-1$ step $2s$ \textbf{parallel}}
				\State $decay \gets exp\left(\frac{t[k+s-1]-t[k+2s-1]}{\tau}\right)$ \Comment{handle unequal time steps}
				\State $x[k+2s-1] \gets x[k+2s-1] + decay*x[k+s-1]$
				
			\EndFor
			\State $s \gets 2s$ 
		\EndWhile
	\EndProcedure
	
	\Procedure{downsweep}{$x \in \mathbb{C}^N$, $t \in \mathbb{R}^N$, $\tau\ \in \mathbb{R}$}
	\Comment{second pass with decreasing step size}
		\State $s \gets N/2$
		\While{$s>1$}    \Comment{ O(log N)}
			\State $s \gets s/2$
			\For{k  $\gets 0$ to $N-1-2s$ step $2s$ \textbf{parallel}}
				\State $decay \gets exp\left(\frac{t[k+2s-1]-t[k+3s-1]}{\tau}\right)$ \Comment{handle unequal time steps}
				\State $x[k+3s-1] \gets x[k+3s-1] + decay*x[k+2s-1]$
			\EndFor
		\EndWhile
	\EndProcedure
	
	\Procedure{sum}{$x \in \mathbb{C}^N$, $t \in \mathbb{R}^N$, $\tau\ \in \mathbb{R}$}
	\State	\Call{upsweep}{$x$, $t$, $\tau$}
	\State 	\Call{downsweep}{$x$, $t$, $\tau$}
	\EndProcedure
	
	\end{algorithmic}

\end{algorithm}



%\begin{algorithm}[H]
%	
%	
%	\KwData{intitial fields F$\el \mathbb{C}^N$}
%	\KwData{arrival times T$\el \mathbb{R}^N$}
%	\KwResult{Intensity at support times}
%
%\end{algorithm}

To calculate the intensity at each support, an modified prefix sum algorithm can be used.



\subsection{Signal to Noise}

\section{Implications for an experimental design}
