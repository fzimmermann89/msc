\chapter{Introduction}
Throughout the history of science, advances in the abilities to image structures at ever-decreasing length scales have enabled break-through discoveries and led to a better understanding of our world.
The optical microscope gave humankind a glance into the world of microorganisms; X-ray crystallography and later on electron microscopy allowed scientists to decipher the building blocks of life\cite{hooke1665,laue1915,ruska1939,watson1953}. These methods simultaneously laid the foundation for modern semiconductor physics and pushed materials science forward
-- which in turn lead to new advancements in imaging \cite{teal1951,hovmoeller1984,jiang2018}.  

In X-ray imaging, the use of synchrotron radiation and, finally, the advent of free-electron lasers (FELs) providing ultra-short and brilliant X-ray pulses pushed the boundaries far beyond what could have been imagined by Röntgen \cite{cloetens1996,emma2010}. One imaging method making use of these capabilities is Coherent Diffractive Imaging (CD). In CDI, spatially and temporally coherent incoming light is scattered within a sample by a fixed phase, conserving its coherence properties and producing an interference pattern which upon measurement can give spatial information with high temporal and spatial resolution \cite{seibert2011,bostedt2010,barke2015}. Incoherent scattering processes, such as fluorescence, are considered background noise worth suppression or filtering in CDI \cite{schultz2013chapter7}.  Recently, Classen et al. suggested instead to \textit{use} the fluorescence for imaging in a new method they termed \textit{Incoherent Diffractive Imaging} (IDI) and is based on the concept of intensity-intensity correlations pioneered by Hanbury Brown and Twiss (HBT) in stellar interferometry \cite{classen2017,hanbury1956}. 

The original  HBT experiments showed that for an incoherent source, the dependence of the correlation between intensities measured with two detectors on their separation encodes spatial information about the source \cite{baym1997}. While in a semi-classical picture this effect can be explained by an averaging out of the instantaneous random phases in the correlation function, the quantum mechanical interpretation has to consider multi-photon states with indistinguishable paths \cite{fano1961,glauber2006}. Thereby the HBT effect famously contradicted the statement by Dirac, that \textit{\enquote{[...] each photon then interferes only with itself. Interference between two different photons never occurs.}} \cite{dirac1958}, and is considered as one of the most influential observations leading to a new understanding of quantum optics and a quantum theory of coherence by Glauber, Mandel et al. \cite{glauber1963,mandel1959, hong1987}.

The main idea behind IDI is leveraging the HBT effect in inner-shell fluorescence for nanoscale imaging: As long as the fluorescence can be recorded in a time interval similar to its coherence time, the phase fluctuation within a single measurement can be considered almost stationary. Thus, averaging over enough pairs of intensities in the far-field and images as separate realizations of the random phases suppresses the random contribution to the phases.  In the multi-photon interference the phase difference resulting from the spatial distribution of the emitters is preserved,  allowing the recovery of the sought-for spatial information about the sample. Distinguishable paths, for example those created by multiple fluorescence energies or arrival time differences greater than the coherence length, will accordingly reduce the contrast of the observed effect \cite{classen2017,trost2020}. 

For many elements, the cross-section of fluorescence exceeds that of coherent scattering by several orders of magnitude, and the strong anisotropy of coherent scattering leads to the need for high dynamic range detectors in CDI \cite{xraylib,attwood1999}.  Fluorescence, on the other hand, can be recorded in all directions, allowing the use of more detectors with less dynamic range and even the integration in existing CDI setups as a second simultaneous imaging method. Furthermore, while wide-angle CDI in a single orientation only gives limited access to a two-dimensional surface inside the three-dimensional reciprocal space, IDI in the same geometry measures a continuous volume inside the reciprocal space \cite{barke2015,classen2017}. In theory, this allows high-resolution three-dimensional imaging from few orientations. Finally, as the emitted fluorescence energies are highly element dependent, the method has an inherent element specificity. Taken together, IDI promises to enhance the imaging capabilities using FELs significantly. 

An already established, related technique, X-ray photon correlation spectroscopy (XPCS), uses intensity fluctuations in coherent X-ray speckle patterns introduced by temporally varying phase shifts to gain insights about dynamics \cite{lehmkuhler2021,grubel2007}. Furthermore, intensity correlation measurements have been used at synchrotrons and FELs to measure the coherence properties of the beam and for scattering experiments in which the incoherence was created by a separate diffusor \cite{yabashi2002,singer2013,inoue2019,gorobtsov2018,schneider2018}.  Nevertheless the only successful imaging experiment thus far that applied intensity correlations to inner shell fluorescence, as proposed by Classen et al., used a metal foil as the sample to measure merely the size of the excitation beam's focal spot \cite{nakumura2020}. No proof-of-principle experiment showing the potential of IDI for high-resolution imaging has yet been published. 

In this thesis, first, the underlying concepts of coherence are recapitulated, leading to a classical interpretation of the HBT effect and a semi-classical study of the signal-to-noise characteristics of IDI. In \fref{chap:simulation}, a method for simulating IDI to validate the theoretical considerations and to identify relevant parameters for an experimental implementation is presented.  Finally, \fref{chap:experiment} presents an experiment conducted at the SACLA FEL facility with the aim of moving IDI beyond just imaging the focal volume to imaging nanoparticles and crystal samples. Here, a description of the sample preparation, the experimental setup, the necessary processing steps of the recorded data, as well as the results, are given.

